\section{Full Orthogonalization Method (FOM)}
\begin{align*}
    \mathcal{K}  & = \mathcal{K}_m(A, \mathbf{r}_0)                                            \\
    \mathcal{L}  & = \mathcal{K}                                                               \\
    \mathbf{x}_m & = \mathbf{x}_0 + V_m \left(V_m^\top A V_m\right)^{-1} V_m^\top \mathbf{r}_0 \\
    V_m          & = [\mathbf{v}_1, \ldots, \mathbf{v}_m] \text{ with } V_m^\top V_m = I
\end{align*}

\begin{enumerate}
    \item How to find an orthogonal basis for $\mathcal{K}_m$?
    \item What is $V_m^\top A V_m$?
    \item When to stop?
          \[ \|\mathbf{r}_m\|_2 \leq \text{tol} \]
\end{enumerate}
\paragraph{1. Arnoldi algorithm}
We can use the Arnoldi algorithm to compute an orthonormal basis for $\mathcal{K}_m(A, \mathbf{r}_0)$.
But what do we get from the Arnoldi algorithm?
\begin{align*}
    V_{m+1}        & = [\mathbf{v}_1, \ldots, \mathbf{v}_{m+1}] \in \mathbb{R}^{n \times (m+1)}, \quad V_m = [\mathbf{v}_1, \ldots, \mathbf{v}_m] \in \mathbb{R}^{n \times m} \\
    \overline{H}_m & = (h_{ij}) \in \mathbb{R}^{(m+1) \times m} \text{ upper Hessenberg matrix}, \qquad H_m := \overline{H}_m(1:m,1:m) \in \mathbb{R}^{m\times m}
\end{align*}
s.t.
\begin{align*}
    A V_m &= V_{m+1} \overline{H}_m = V_m H_m + h_{m+1,m} \mathbf{v}_{m+1}\mathbf{e}_m^\top\\
    H_m   &= V_m^\top A V_m
\end{align*}

Using the Galerkin condition for FOM (take $\mathcal{L}_m =\mathcal{K}_m$) we obtain the small system
\[
    H_m \mathbf{y}_m = V_m^\top \mathbf{r}_0 = \beta \mathbf{e}_1, \qquad \beta=\|\mathbf{r}_0\|_2,
\]
so
\[
    \mathbf{x}_m = \mathbf{x}_0 + V_m \mathbf{y}_m, \qquad \mathbf{y}_m = H_m^{-1} (\beta \mathbf{e}_1).
\]

The residual can be computed cheaply from the Arnoldi relation:
\begin{align*}
    \mathbf{r}_m &= \mathbf{r}_0 - A V_m \mathbf{y}_m
    = \beta \mathbf{v}_1 - V_{m+1}\overline{H}_m \mathbf{y}_m \\
    &= \beta \mathbf{v}_1 - V_m H_m \mathbf{y}_m - h_{m+1,m}\mathbf{v}_{m+1} \mathbf{e}_m^\top \mathbf{y}_m
    = -h_{m+1,m}\mathbf{v}_{m+1} \mathbf{e}_m^\top \mathbf{y}_m,
\end{align*}
since $H_m\mathbf{y}_m=\beta\mathbf{e}_1$. Hence
\[
    \|\mathbf{r}_m\|_2 = |h_{m+1,m}|\,|\mathbf{e}_m^\top \mathbf{y}_m|.
\]

Thus we get the FOM algorithm (Arnoldi performed incrementally; solve the small system at each step and check residual):

\begin{algorithm}[htbp]
    \begin{algorithmic}
        \Require
        \State $A \in \mathbb{R}^{n \times n}, \mathbf{b} \in \mathbb{R}^n, \mathbf{x}_0 \in \mathbb{R}^n, m_{\max} \in \mathbb{N}, \text{tol} > 0$
        \State $\mathbf{r}_0 = \mathbf{b} - A\mathbf{x}_0, \quad \beta = \|\mathbf{r}_0\|_2$
        \State $\mathbf{v}_1 = \mathbf{r}_0 / \beta$
        \Ensure
        \State $\mathbf{x}_j$ approximations, stop when converged or breakdown
        \For{$j = 1, 2, \ldots, m_{\max}$}
        \State Perform one Arnoldi step to compute $h_{1:j+1,j}$ and $\mathbf{v}_{j+1}$ (see Alg. \ref{alg:arnoldi-lecture})
        \State Let $H_j = \overline{H}_j(1:j,1:j)$ and $V_j = [\mathbf{v}_1,\ldots,\mathbf{v}_j]$
        \State Solve $H_j \mathbf{y}_j = \beta \mathbf{e}_1$
        \State $\mathbf{x}_j = \mathbf{x}_0 + V_j \mathbf{y}_j$
        \State $\mathbf{r}_j = -h_{j+1,j} \mathbf{v}_{j+1} \mathbf{e}_j^\top \mathbf{y}_j$
        \If{$\|\mathbf{r}_j\|_2 \leq \text{tol}$}
        \State Return $\mathbf{x}_j$
        \EndIf
        \If{$h_{j+1,j}=0$}
        \State Breakdown: exact solution in $\mathcal{K}_j$ (stop)
        \EndIf
        \EndFor
    \end{algorithmic}
    \caption{Full Orthogonalization Method (FOM)}
    \label{alg:fom}
\end{algorithm}