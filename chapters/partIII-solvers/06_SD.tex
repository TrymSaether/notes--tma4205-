\documentclass[../../main.tex]{subfiles}
\begin{document}
\chapter{Steepest Descent and 1D Projection Methods}\label{ch:steepest-descent}
\section{Notation}
We wish to solve the linear system
\[
    A \mathbf{x} = \mathbf{b}, \quad A \in \mathbb{R}^{n \times n}, \quad \mathbf{x}, \mathbf{b} \in \mathbb{R}^n
\]

\subsection{Quadratic form}
The quadratic form is just a scalar, quadratic function $f: \mathbb{R}^n \to \mathbb{R}$ that maps a vector $\mathbf{x}$ to a scalar:
\begin{equation}\label{eq:quadratic-form}
    f(\mathbf{x}) = \frac12 \mathbf{x}^{\top} A \mathbf{x} - \mathbf{b}^{\top} \mathbf{x} + c
\end{equation}
where $c$ is just a constant.

\paragraph{Gradient}
Now we want to observe how the function $f$ changes as we change $\mathbf{x}$, i.e. we want to compute the gradient $\nabla f(\mathbf{x})$.
\begin{align}
    \nabla f(\mathbf{x}) & = \begin{bmatrix}
                                 \frac{\partial}{\partial x_1} f(\mathbf{x}) \\
                                 \vdots                                      \\
                                 \frac{\partial}{\partial x_n} f(\mathbf{x})
                             \end{bmatrix}                        \\
                         & = \begin{bmatrix}
                                 \frac12 (2 a_{11} x_1 + a_{12} x_2 + \ldots + a_{1n} x_n) - b_1 \\
                                 \vdots                                                          \\
                                 \frac12 (a_{n1} x_1 + a_{n2} x_2 + \ldots + 2 a_{nn} x_n) - b_n
                             \end{bmatrix}    \\
                         & = \frac12 (A + A^{\top}) \mathbf{x} - \mathbf{b} \label{eq:gradient}
\end{align}

Now, if we assume that $A$ is symmetric ($A = A^{\top}$), then \eqref{eq:gradient} simplifies to
\begin{equation}
    \nabla f(\mathbf{x}) = A \mathbf{x} - \mathbf{b}
\end{equation}
Now, under the assumption that $A$ also is positive definite ($A \succ 0$), the quadratic form \eqref{eq:quadratic-form} is strictly convex and has a unique minimizer $\mathbf{x}_{\star}$.
\begin{equation}
    \nabla f(\mathbf{x}_{\star}) = A \mathbf{x}_{\star} - \mathbf{b} = 0 \implies A \mathbf{x}_{\star} = \mathbf{b}
\end{equation}


\textbf{Case 1:} Suppose $A$ is \emph{SPD}, then the quadratic form \eqref{eq:quadratic-form} is strictly convex and has a unique minimizer $\mathbf{x}_{\star}$ which is equivalent to the solution of the linear system \eqref{eq:linear-system}.

\textbf{Case 2:} Suppose $A$ is \emph{symmetric} (be it positive definite or not).
Let $\mathbf{x}$ be a point that satisfies \eqref{eq:linear-system} and minimizes the quadratic form \eqref{eq:quadratic-form}.
Now we introduce the error term $\mathbf{e}$ s.t.
\begin{align}
    f(\mathbf{x} + \mathbf{e}) & = \dfrac12 (\mathbf{x} + \mathbf{e})^{\top} A (\mathbf{x} + \mathbf{e}) - \mathbf{b}^{\top} (\mathbf{x} + \mathbf{e}) + c                                                                                                                   \\
                               & = \overbrace{\dfrac12 \mathbf{x}^{\top} A \mathbf{x} - \mathbf{b}^{\top} \mathbf{x} + c}^{f(\mathbf{x})} +  \dfrac12 \mathbf{e}^{\top} A \mathbf{e} + \mathbf{e}^{\top}\overbrace{A\mathbf{x}}^{\mathbf{b}}  - \mathbf{b}^{\top} \mathbf{e} \\
                               & = f(\mathbf{x}) + \dfrac12 \mathbf{e}^{\top} A \mathbf{e} + \mathbf{e}^{\top} \mathbf{b} - \mathbf{b}^{\top} \mathbf{e}                                                                                                                     \\
                               & = f(\mathbf{x}) + \dfrac12 \mathbf{e}^{\top} A \mathbf{e}
\end{align}
If $A$ is positive definite, then $\mathbf{e}^{\top} A \mathbf{e} > 0$ for all $\mathbf{e} \neq 0$, and therefore $\mathbf{x}$ is the unique minimizer of $f$.

Now if $A$ is not SPD, then the equation \eqref{eq:gradient} hints that we try to find the solution of the system $\frac12 (A + A^{\top}) \mathbf{x} = \mathbf{b}$ instead, which we will discuss later.

Throughout the first section, we will use the following $2\times 2$ SPD example to illustrate and explain the concepts and methods we introduce.

\begin{example}{SPD problem}{spd-problem}
    \[
        A =
        \begin{bmatrix}
            3 & 2 \\
            2 & 6
        \end{bmatrix}, \quad \mathbf{b} =
        \begin{bmatrix}
            2 \\
            -8
        \end{bmatrix},
        \quad
        c = 0.
    \]
    The corresponding quadratic form is
    \[
        f(\mathbf{x}) = \frac12 \mathbf{x}^{\top} A \mathbf{x} - \mathbf{b}^{\top} \mathbf{x} = \frac12 (3x_1^2 + 2x_1 x_2 + 6x_2^2) - (2x_1 - 8x_2).
    \]
    The solution/minimizer is
    \[
        \mathbf{x}_{\star} =
        \begin{bmatrix}
            2 \\
            -2
        \end{bmatrix}.
    \]
\end{example}

If we plot the corresponding quadratic form of this problem, we get the following surface plot:

\begin{figure}[H]
    \centering
    \begin{tikzpicture}
    \begin{groupplot}[
            group style={group size=2 by 1, horizontal sep=1.5cm},
            width=0.48\textwidth,
            height=6cm,
            colormap/cool,
            xmin=-6, xmax=6,
            ymin=-4, ymax=4,
        ]
        \nextgroupplot[
            view={20}{30},
            xlabel={$x_1$},
            ylabel={$x_2$},
            zlabel={$f(\mathbf{x})$},
            grid=major,
        ]
        \addplot3[
            surf,
            domain=-6:6,
            domain y=-4:4,
            samples=30,
        ]
        {0.5 * (3 * x^2 + 4 * x * y + 6 * y^2) - (2 * x - 8 * y)};
        \addplot3[
            mark=x,
            only marks,
            color=red,
            mark size=3pt
        ]
        coordinates {(2, -2, -10)};
        \nextgroupplot[
            xlabel={$x_1$},
            ylabel={$x_2$},
            view={0}{90},
            grid=major,
        ]
        \addplot3[
            contour filled={number=30,labels=false},
            samples=60,
            domain=-6:6,
            domain y=-4:4,
        ]
        {0.5 * (3 * x^2 + 4 * x * y + 6 * y^2) - (2 * x - 8 * y)};
        % Gradient
        \addplot3[
            quiver={
                    u={3*x + 2*y - 2},
                    v={2*x + 6*y + 8},
                    w={0},
                    scale arrows=0.1,
                    every arrow/.append style={-Stealth, line width=0.5pt, color=black}
                },
            domain=-6:6,
            domain y=-4:4,
            samples=12,
        ]
        {0};
        % Mark the minimizer
        \addplot3[
            mark=x,
            only marks,
            color=red,
            mark size=3pt
        ]
        coordinates {(2, -2, -10)};
    \end{groupplot}
\end{tikzpicture}
    \caption{\textbf{(a)} We see the surface plot of the quadratic form $f$ curving upwards, indicating that it is convex and therefore has a minimum.
        \textbf{(b)} Same plot seen from above, with gradient vectors added. The gradient vectors point in the direction of steepest ascent, and we see that they point away from the minimizer (red cross).}
    \label{fig:quadratic-form}
\end{figure}


Let $A$ be SPD, with a given initial guess $\mathbf{x}_0$, and residual $\mathbf{r}_0 = \mathbf{b} - A \mathbf{x}_0$.

\begin{align*}
    \mathcal{L}              & = \mathcal{K} = \operatorname{span}\{\mathbf{r}\}                                                                                                              \\
    \mathbf{x}_{k+1}         & = \mathbf{x}_k + \alpha_k \mathbf{r}_k, \quad \alpha_k = \|\mathbf{r}_k\|_2^2 / \mathbf{r}_k^{\top} A \mathbf{r}_k                                             \\
    \mathbf{d}_k             & = \mathbf{x}_{\star} - \mathbf{x}_k                                                                                                                            \\
    \|\mathbf{d}_{k+1}\|_A   & \leq \|\mathbf{d}_k\|_A                                                                                                                                        \\
    \|\mathbf{d}_{k+1}\|_A^2 & = \|\mathbf{d}_k\|_A^2\left(1 - \frac{(\mathbf{r}_k^{\top} \mathbf{r}_k)^2}{\mathbf{r}_k^{\top} A \mathbf{r}_k \mathbf{r}_k^{\top} A^{-1} \mathbf{r}_k}\right)
\end{align*}

Let $B \in \mathbb{R}^{n \times n}$ be SPD, and using Kantorovich's inequality \ref{thm:kantorovich} then for all $\mathbf{x} \in \mathbb{R}^n$
\[
    \frac{\|\mathbf{x}\|_B^2 \|\mathbf{x}\|_{B^{-1}}^2}{\|\mathbf{x}\|_2^4} \leq \frac{1}{4}\cdot\frac{(\lambda_1 + \lambda_n)^2}{\lambda_1 \lambda_n}, \qquad \lambda_1 \geq \ldots \geq \lambda_n > 0
\]

$B$ is SPD so there exists $Q$ orthogonal and $\Lambda = \operatorname{diag}(\lambda_1, \ldots, \lambda_n)$ such that $B = Q^{\top}\Lambda Q$. Choose $\|\mathbf{x}\|_2 = 1$ where $\|Q \mathbf{x}\|_2 = \|\mathbf{x}\|_2 = 1$. Then:
\begin{align*}
    B^{-1}                    & = Q^{\top} \Lambda^{-1} Q                                                                                                           \\
    \|\mathbf{x}\|_B^2        & = \mathbf{x}^{\top} B \mathbf{x} = (Q\mathbf{x})^{\top} \Lambda (Q\mathbf{x}) = \sum_{i=1}^n \lambda_i y_i^2, \quad y = Q\mathbf{x} \\
    \|\mathbf{x}\|_{B^{-1}}^2 & = \mathbf{x}^{\top} B^{-1} \mathbf{x} = (Q\mathbf{x})^{\top} \Lambda^{-1} (Q\mathbf{x}) = \sum_{i=1}^n \lambda_i^{-1} y_i^2         \\
\end{align*}
$(\overline{\lambda}, \overline{\lambda}^{-1})$ as a weighted discre center of gravity for the point $(\lambda_i, \frac{1}{\lambda_i})$ for $i = 1, \ldots, n$.

\[ \ell(\lambda) = \frac{1}{\lambda_1} + \frac{1}{\lambda_n} - \frac{\lambda}{\lambda_1 \lambda_n}, \qquad \ell(\lambda_1) = \frac{1}{\lambda_1}, \qquad \ell(\lambda_n) = \frac{1}{\lambda_n} \]

Then $(\bar\lambda, \bar{\lambda}^{-1})$ is below $\ell(\lambda)$:
\[
    \bar\lambda^{-1} \leq \ell(\bar\lambda)
\]
which has maximum at $\lambda = \tfrac12(\lambda_1 + \lambda_n)$.
\[
    \bar\lambda \bar\lambda^{-1} \leq \frac{(\lambda_1 + \lambda_n)^2}{4 \lambda_1 \lambda_n} = \bar\lambda\left(\frac{1}{\lambda_1} + \frac{1}{\lambda_n}\right)
\]


If $A$ has the eigenvalues $0 < \lambda_1 \leq \ldots \leq \lambda_n$, then:
\begin{align*}
    \frac{\|\mathbf{r}_k\|_2^4}{\|\mathbf{r}_k\|_A^2 \|\mathbf{r}_k\|_{A^{-1}}^2} & \geq \frac{4 \lambda_1 \lambda_n}{(\lambda_1 + \lambda_n)^2}                                      \\
    \|\mathbf{d}_{k+1}\|_A^2                                                      & \leq \|\mathbf{d}_k\|_A^2\left(1 - 4 \frac{\lambda_1 \lambda_n}{(\lambda_1 + \lambda_n)^2}\right) \\
                                                                                  & = \|\mathbf{d}_k\|_A^2\left(\frac{\lambda_n - \lambda_1}{\lambda_n + \lambda_1}\right)^2
\end{align*}

\begin{example}{Discrete Laplacian in 2D}{}
    \begin{align*}
        A & =
        \begin{bmatrix}
            B  & -I &        &        & 0  \\
            -I & B  & -I     &        &    \\
               & -I & \ddots & \ddots &    \\
               &    & \ddots & \ddots & -I \\
            0  &    &        & -I     & B
        \end{bmatrix} \in \mathbb{R}^{N^2 \times N^2},
        \begin{bmatrix}
            4  & -1 &        & 0 \\
            -1 & 4  & -1     &   \\
               & -1 & \ddots &   \\
            0  &    &        & 4
        \end{bmatrix} \in \mathbb{R}^{N \times N} \\
    \end{align*}
    Eigenvalues of $A$:
    \[
        \lambda_{ij} = 4 - 2\left(\cos\left(\frac{i \pi}{N+1}\right) + \cos\left(\frac{j \pi}{N+1}\right)\right), \quad i, j = 1, \ldots, N
    \]
    \begin{align*}
        \lambda_{\max}                                                          & = 4 \text{ if } N \text{ odd}                                                                                                                \\
        \lambda_{\min}                                                          & = 4 - 4\cos\left(\frac{\pi}{N+1}\right)                                                                                                      \\
        \frac{\lambda_{\max} - \lambda_{\min}}{\lambda_{\max} + \lambda_{\min}} & = \frac{4\cos\left(\frac{\pi}{N+1}\right)}{8 - 4\cos\left(\frac{\pi}{N+1}\right)} \approx 1 - \frac12\left(\frac{\pi}{N+1}\right)^2 + \ldots \\
    \end{align*}
    So for $N$ large, convergence is slow.
\end{example}

\section{Other One-Dimensional Projection Methods}
Let $\mathcal{K} = \operatorname{span}\{\mathbf{v}\}$, $\mathcal{L} = \operatorname{span}\{\mathbf{w}\}$.
One step, starting from $\mathbf{x}_0$:
\begin{align*}
    \tilde{\mathbf{x}} & = \mathbf{x}_0 + \alpha \mathbf{v}, \quad \alpha = \frac{\mathbf{w}^{\top} \mathbf{r}_0}{\mathbf{w}^{\top} A \mathbf{v}} \\
    \tilde{\mathbf{r}} & = \mathbf{b} - A\tilde{\mathbf{x}} = \mathbf{r}_0 - \alpha A \mathbf{v}
\end{align*}
if SD: $\mathbf{v} = \mathbf{w} = \mathbf{r}_0$.


\begin{example}{}{}
    If $A$ is SPD, with $\mathcal{L} = \mathcal{K} = \operatorname{span}\{\mathbf{r}_k\}$, then:
    \begin{align*}
        \mathbf{x}_{k+1} & = \mathbf{x}_k + \alpha_k \mathbf{r}_k, \quad \alpha_k  \in  \mathbb{R}                               \\
        \mathbf{r}_{k+1} & = \mathbf{b} - A \mathbf{x}_{k+1} = \mathbf{r}_k - \alpha_k A \mathbf{r}_k                            \\
        \mathbf{r}_{k+1} & \perp \mathbf{r}_k \Rightarrow \mathbf{r}_k^{\top} (\mathbf{r}_k - \alpha_k A \mathbf{r}_k) = 0 \quad
        \Rightarrow \alpha_k = \frac{\mathbf{r}_k^{\top} \mathbf{r}_k}{\mathbf{r}_k^{\top} A \mathbf{r}_k}                       \\
        \mathbf{d}_k     & = \mathbf{x}_\star - \mathbf{x}_k                                                                     \\
        \mathbf{r}_k     & = \mathbf{b} - A \mathbf{x}_k = A \mathbf{x}_\star - A \mathbf{x}_k = A \mathbf{d}_k                  \\
    \end{align*}

    We want to estimate $\|d_{k+1}\|_A \leq C \|d_k\|_A$ for some $C < 1$.
    \begin{align*}
        \mathbf{r}_{k+1}         & = \mathbf{b} - A \mathbf{x}_{k+1} = A(\mathbf{x}_\star - \mathbf{x}_{k+1}) = A \mathbf{d}_{k+1} = A \mathbf{d}_k - \alpha_k A \mathbf{r}_k      \\
        \mathbf{d}_{k+1}         & = \mathbf{d}_{k+1}^{\top} A \mathbf{d}_{k+1} = \mathbf{d}_{k+1}^{\top} \mathbf{r}_{k+1}                                                         \\
                                 & = (\mathbf{d}_k - \alpha_k \mathbf{r}_k)^{\top} \mathbf{r}_{k+1} = \mathbf{d}_k^{\top} \mathbf{r}_{k+1}                                         \\
                                 & = \mathbf{d}_k^{\top} (\mathbf{r}_k - \alpha_k A \mathbf{r}_k) = \mathbf{d}_k^{\top} \mathbf{r}_k - \alpha_k \mathbf{d}_k^{\top} A \mathbf{r}_k \\
                                 & = \mathbf{d}_k^{\top} A \mathbf{d}_k - \alpha_k \mathbf{r}_k^{\top} \mathbf{r}_k                                                                \\
                                 & = \|\mathbf{d}_k\|_A^2 - \alpha_k \|\mathbf{r}_k\|^2                                                                                            \\
                                 & = \|\mathbf{d}_k\|_A^2 - \frac{\|\mathbf{r}_k\|^4}{\|\mathbf{r}_k\|_A^2}                                                                        \\
        \|\mathbf{d}_{k+1}\|_A^2 & = \|\mathbf{d}_k\|_A^2\left(1 - \frac{\|\mathbf{r}_k\|^4}{\|\mathbf{r}_k\|_A^2 \|\mathbf{r}_k\|_{A^{-1}}^2}\right)
    \end{align*}
\end{example}

\end{document}
