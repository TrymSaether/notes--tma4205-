% =============================
% PREAMBLE — LuaLaTeX (lean, NO/only)
% =============================

% Quiet harmless warnings (optional)
\usepackage{silence}
\WarningFilter{latex}{Command \showhyphens has changed}
\WarningFilter{latexfont}{Font shape}

% --- Language & fonts ---
\usepackage[english,shorthands=off]{babel}
\usepackage{fontspec}
\usepackage{amsmath,amsthm,mathtools}
\mathtoolsset{showonlyrefs,centercolon}
\usepackage[mathrm=sym,  warnings-off={mathtools-colon, mathtools-overbracket}]{unicode-math}

% Text fonts
\setmainfont{Fira Sans}
\setsansfont{IBM Plex Sans Condensed}
\setmonofont{Fira Code}
\setmathfont{Fira Math}
\setmathfont{STIX Two Math}[range={\mathscr,\mathbb,\mathfrak,\mathcal,\setminus,\Re,\Im,\triangleright,\top,\ddots,\vdots,\star,\bigcup},Scale=MatchUppercase]

% --- Typography & lists ---
\usepackage[final]{microtype}
\UseMicrotypeSet[protrusion]{basicmath}
\microtypesetup{tracking=true,expansion=true}
\usepackage[table]{xcolor} % table colors without separate colortbl
\usepackage{enumitem}
\setlist{nosep}
\usepackage[autostyle]{csquotes}

% --- Structure & graphics ---
\usepackage{subfiles}                    % fast per-chapter compilation
\usepackage{booktabs,tabularx,array,multirow}
\usepackage{siunitx}
\sisetup{detect-all,group-minimum-digits=4,output-decimal-marker=.}
\usepackage{float}
\usepackage{subcaption}

% --- Layout, colors, boxes ---
% ==============================
% COLORS
% ==============================

% --- Semantic environment colors  ---
\definecolor{def-color}{HTML}{E40125}   % red (NTNU red)
\definecolor{thm-color}{HTML}{1346E4}   % deep blue
\definecolor{lem-color}{HTML}{05C4D9}   % cyan
\definecolor{exm-color}{HTML}{21340A}   % dark olive
\definecolor{rem-color}{HTML}{18B640}   % green
\definecolor{prop-color}{HTML}{9C27B0}  % purple
\definecolor{cor-color}{HTML}{FF5722}   % orange
\definecolor{alg-color}{HTML}{37474F}   % blue-grey
\definecolor{mot-color}{HTML}{9C27B0}   % reuse purple

% --- Utility accents (optional) ---
\definecolor{filing-color}{HTML}{607D8B}
\definecolor{railing-color}{HTML}{795548}
\definecolor{flag-color}{HTML}{009688}
\definecolor{spaceblack}{HTML}{403E3D}

% --- Soft yes/no backgrounds ---
\definecolor{yescolor}{RGB}{200,255,200}
\definecolor{nocolor}{RGB}{255,200,200}
\definecolor{headerblue}{RGB}{225,238,255}

% --- NTNU palette (reference only; prefer semantic names above) ---
\definecolor{ntnu-blue}{HTML}{00509E}
\definecolor{ntnu-red}{HTML}{E40125}
\definecolor{ntnu-green}{HTML}{BCD025}
\definecolor{ntnu-lightblue}{HTML}{6096D0}
\definecolor{ntnu-orange}{HTML}{EF8114}
\definecolor{ntnu-sand}{HTML}{CFB887}
\definecolor{ntnu-magenta}{HTML}{B01B81}
\definecolor{ntnu-yellow}{HTML}{F7D019}
\definecolor{ntnu-purple}{HTML}{482776}
\definecolor{ntnu-turquoise}{HTML}{3CBFBE}

% --- State colors (kept minimal; avoid near-duplicates) ---
\definecolor{state-yellow}{HTML}{F7D019}
\definecolor{state-orange}{HTML}{FF6F00}
\definecolor{state-gray}{HTML}{B0BEC5}
\colorlet{state-blue}{thm-color!75}
\colorlet{state-red}{red!80}

% --- Hyperref colors (hooked in preamble’s \hypersetup) ---
\definecolor{Link}{HTML}{1346E4} % align with thm-color (consistent brand)
\definecolor{Cite}{HTML}{0F766E} % teal 700 (better contrast)
\definecolor{URL}{HTML}{0E7490}  % cyan 700
\definecolor{File}{HTML}{B45309} % amber 700

% --- Auto tints (use these for colback; full colors for colframe) ---
\colorlet{def-tint}{def-color!8}
\colorlet{thm-tint}{thm-color!8}
\colorlet{lem-tint}{lem-color!10}
\colorlet{exm-tint}{exm-color!6}
\colorlet{rem-tint}{rem-color!10}
\colorlet{prop-tint}{prop-color!8}
\colorlet{cor-tint}{cor-color!8}
\colorlet{alg-tint}{alg-color!8}
\colorlet{mot-tint}{mot-color!8}
\colorlet{ntnu-yellow-tint}{ntnu-yellow!12}
\usepackage[most]{tcolorbox}
\usepackage{algorithm}
\usepackage[noend]{algpseudocode}
\usepackage[headheight=15pt,margin=2.5cm,includeheadfoot]{geometry}
\usepackage{parskip}

\usepackage{fancyhdr}
\pagestyle{fancy}
\fancyhf{}
\fancyhead[L]{\nouppercase{\leftmark}}
\fancyhead[R]{\thepage}
\renewcommand{\headrulewidth}{0.4pt}
\renewcommand{\footrulewidth}{0pt}

% --- Links & refs (keep order) ---
\usepackage[colorlinks,plainpages=false]{hyperref}
\hypersetup{
    colorlinks=true,
    linkcolor=thm-color,
    citecolor=thm-color,
    urlcolor=cyan,
    filecolor=magenta
}
\usepackage[capitalize,nameinlink,noabbrev]{cleveref}

% --- Your split macros (formatting vs notation) ---
% ==============================
% COMMANDS — formatting & convenience (non-semantic)
% ==============================

% Mathtools paired delimiters (with starred auto-size)
\DeclarePairedDelimiter{\abs}{\lvert}{\rvert}
\DeclarePairedDelimiter{\norm}{\lVert}{\rVert}
\DeclarePairedDelimiter{\set}{\lbrace}{\rbrace}

% Inner product with comma spacing
\DeclarePairedDelimiterX{\inner}[2]{\langle}{\rangle}{#1,\,#2}

% Derivatives (clear, flexible)
\NewDocumentCommand{\dv}{o m}{\frac{d\,#1}{d #2}}            % \dv[f]{x}
\NewDocumentCommand{\dvn}{m o m}{\frac{d^{#1}\,#2}{d #3^{#1}}} % \dvn{2}[f]{x}
\NewDocumentCommand{\pdv}{o m}{\frac{\partial\,#1}{\partial #2}}
\NewDocumentCommand{\pdvn}{m o m}{\frac{\partial^{#1}\,#2}{\partial #3^{#1}}}

% Common text shortcuts (optional; comment out if you prefer)
\newcommand{\ie}{i.e.\ }
\newcommand{\eg}{e.g.\ }
\newcommand{\aka}{aka.\ }
\newcommand{\st}{\text{s.t.}\ }   % formatting/convenience macros
% ==============================
% NOTATION — semantic symbols & operators
% ==============================

% Sets & fields
\newcommand{\N}{\mathbb{N}}
\newcommand{\Z}{\mathbb{Z}}
\newcommand{\Q}{\mathbb{Q}}
\newcommand{\R}{\mathbb{R}}
\newcommand{\C}{\mathbb{C}}
\newcommand{\F}{\mathbb{F}}

% Linear algebra operators
\DeclareMathOperator{\diag}{diag}
\DeclareMathOperator{\tr}{tr}
\DeclareMathOperator{\rank}{rank}
\DeclareMathOperator{\adj}{adj}
\DeclareMathOperator{\col}{col}
\DeclareMathOperator{\row}{row}
\DeclareMathOperator{\range}{range}
\DeclareMathOperator{\Span}{span}
\DeclareMathOperator{\Ker}{ker}
\DeclareMathOperator{\Img}{im}

% Optimization
\DeclareMathOperator*{\argmin}{arg\,min}
\DeclareMathOperator*{\argmax}{arg\,max}

% Probability/statistics (optional)
\DeclareMathOperator{\Var}{Var}
\DeclareMathOperator{\Cov}{Cov}
\newcommand{\E}{\mathbb{E}}

% Numbering
\numberwithin{equation}{chapter}
   % semantic symbols/sets/operators

% --- Theorem environments ---
\tcbset{
    base theorem style/.style={
            enhanced, breakable, sharp corners,
            fonttitle=\bfseries,
            before skip=10pt, after skip=10pt,
            separator sign=.
        },
}

\newtcbtheorem[number within=chapter]{definition}{Definition}{%
    enhanced jigsaw,
    colback=def-color!10, colframe=def-color!80!black, coltitle=black,
    fonttitle=\bfseries,
    attach boxed title to top left={xshift=10pt,yshift=-\tcboxedtitleheight/2},
    boxed title style={colback=def-color!10,colframe=def-color!80!black,height=16pt,valign=center,bean arc},
    label separator={}, top=8pt,bottom=2pt,left=4pt,right=4pt,boxrule=1pt
}{def:}

\newtcbtheorem[use counter from=definition]{theorem}{Theorem}{%
    enhanced jigsaw,
    colback=thm-color!10, colframe=thm-color!80!black, coltitle=black,
    fonttitle=\bfseries,
    attach boxed title to top left={xshift=10pt,yshift=-\tcboxedtitleheight/2},
    boxed title style={colback=thm-color!10,colframe=thm-color!80!black,height=16pt,valign=center,bean arc},
    label separator={}, top=8pt,bottom=2pt,left=4pt,right=4pt,boxrule=1pt
}{thm:}

\newtcbtheorem{lemma}{Lemma}{%
    enhanced jigsaw,
    frame hidden, boxrule=0pt,
    colback=lem-color!10, colframe=lem-color!10,
    fonttitle=\sffamily\bfseries,
    attach boxed title to top left={yshift=-\tcboxedtitleheight},
    boxed title style={boxrule=0pt,boxsep=2pt,interior code={\fill[lem-color, smooth]
                    (interior.north west)--(interior.south west)--([xshift=-2mm]interior.south east)--
                    ([xshift=2mm]interior.north east)--cycle;}},
    label separator={}, borderline north={1pt}{0pt}{lem-color},
    before upper={\hspace{\tcboxedtitlewidth}},
    sharp corners, top=2pt,bottom=2pt,left=4pt,right=4pt
}{lem}

\newtcbtheorem{example}{Example}{%
    base theorem style,
    colback=white, colframe=white,
    borderline west={1.5pt}{0pt}{exm-color},
    left=8pt, coltitle=exm-color, boxsep=2pt, top=0pt, bottom=0pt
}{ex}

\NewTColorBox{solution}{ O{} }{%
    enhanced, colback=white, colframe=white,
    borderline east={1.5pt}{0pt}{rem-color},
    left=8pt,right=4pt,top=4pt,bottom=4pt,
    before skip=10pt,after skip=10pt,breakable,sharp corners,boxrule=0pt,frame hidden,parbox=false,
    overlay={\node[rotate=-90, anchor=south west,font=\tiny\bfseries]
            at ([yshift=3pt, xshift=-2pt]frame.north east) {\MakeUppercase{Løsning}};}
}

\newtcbtheorem{remark}{Remark}{%
    base theorem style, colback=white, colframe=white,
    borderline west={1.5pt}{0pt}{rem-color},
    left=8pt, top=0pt, bottom=0pt, boxsep=2pt, coltitle=black
}{rem}

\newtcbtheorem{proposition}{Proposition}{%
    enhanced jigsaw,
    colback=prop-color!10, colframe=prop-color!80!black,
    fonttitle=\bfseries,
    attach boxed title to top left={yshift=-\tcboxedtitleheight},
    boxed title style={boxrule=0pt,boxsep=2.5pt,colback=prop-color!80!black,
            colframe=prop-color!80!black,sharp corners=uphill},
    label separator={}, top=\tcboxedtitleheight,bottom=2pt,left=2pt,right=2pt,
    before skip=10pt,after skip=10pt,breakable
}{prop}

\newtcbtheorem{corollary}{Corollary}{%
    enhanced jigsaw,
    colback=cor-color!10, colframe=cor-color!80!black, boxrule=0pt,
    fonttitle=\sffamily\bfseries, coltitle=black,
    label separator={},
    description font=\normalfont\sffamily, description delimiters={(}{)},
    attach title to upper, after title={.\ },
    frame hidden, borderline west={2pt}{0pt}{cor-color},
    sharp corners, top=2pt,bottom=2pt,left=5pt,right=5pt,
    before skip=10pt,after skip=10pt,breakable
}{cor}

\newtcbtheorem{summary}{Summary}{%
    enhanced jigsaw,
    colback=ntnu-yellow!10, colframe=ntnu-yellow!80!black,
    coltitle=black, fonttitle=\bfseries,
    attach boxed title to top left={xshift=10pt,yshift=-\tcboxedtitleheight/2},
    boxed title style={colback=ntnu-yellow!10,colframe=ntnu-yellow!80!black,height=16pt,bean arc},
    label separator={}, sharp corners, top=6pt,bottom=2pt,left=2pt,right=2pt,
    before skip=10pt,after skip=10pt,breakable
}{sum}

\renewenvironment{proof}[1][\proofname]{%
    \begin{tcolorbox}[blanker,borderline west={2pt}{0pt}{thm-color},
            before skip=10pt, after skip=10pt, left=8pt, breakable]%
        \textbf{#1}. }{\,\hfill\qedsymbol\end{tcolorbox}}

% --- TikZ / pgfplots ---
\usepackage{standalone}
\usepackage{pgfplots,tikz}
\pgfplotsset{
    compat=newest,
    tick label style={/pgf/number format/fixed},
    scaled ticks=false
}
\usetikzlibrary{
    arrows.meta,calc,positioning,backgrounds,patterns,decorations.pathreplacing,
    3d
}
\tikzset{>=Latex} % nice arrowheads

\pgfplotsset{compat=newest}
