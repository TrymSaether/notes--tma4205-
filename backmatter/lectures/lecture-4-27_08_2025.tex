\section{Lecture 4: 27.08.2025}

\subsection{Similarity and eigenvectors}

Let \(A\in\mathbb{C}^{n\times n}\). If \(B=X^{-1}AX\) with \(\det X\neq0\), then \(A\) and \(B\) are similar and have the same eigenvalues. If \(Av=\lambda v\), then \(X^{-1}v\) is an eigenvector of \(B\) with eigenvalue \(\lambda\).

If \(A\) is diagonalizable with eigenbasis \(V=[v_1,\ldots,v_n]\), then
\[
V^{-1}AV=\Lambda=\operatorname{diag}(\lambda_1,\ldots,\lambda_n).
\]
If \(A\) is defective, there exists invertible \(X\) with
\[
X^{-1}AX=J=\text{blockdiag}\Big(J_1(\lambda_1),\ldots,J_s(\lambda_s)\Big),\qquad
J_k(\lambda)=\begin{bmatrix}\lambda&1\\[2pt]0&\lambda\end{bmatrix}\ \text{or larger}.
\]

\subsection{Schur decomposition}
\begin{theorem}{Schur decomposition}
For any \(A\in\mathbb{C}^{n\times n}\) there exists a unitary \(Q\) and upper triangular \(T\) such that
\[
A=QTQ^H,\qquad T=Q^HAQ.
\]
\end{theorem}

\begin{proof}
Pick a unit eigenvector \(u\) of \(A\), complete to a unitary \(U=[u\ \tilde U]\). Then
\[
U^HAU=\begin{bmatrix}\alpha&c^H\\0&\tilde A\end{bmatrix}.
\]
By induction, choose unitary \(\tilde V\) with \(\tilde V^H\tilde A\tilde V=T_{n-1}\). With
\(Q=U\,\operatorname{diag}(1,\tilde V)\),
\[
Q^HAQ=\begin{bmatrix}\alpha&b^H\\0&T_{n-1}\end{bmatrix},
\]
which is upper triangular.\qed
\end{proof}

\paragraph{Hermitian case:}
If \(A=A^H\), then \(T\) is normal and upper triangular, hence diagonal with real entries. Thus
\[
A=Q\Lambda Q^H,\qquad \Lambda=\operatorname{diag}(\lambda_1,\ldots,\lambda_n)\in\mathbb{R}^{n\times n}.
\]

\subsection{Real Schur form}
For \(A\in\mathbb{R}^{n\times n}\) there exists orthogonal \(Q\) with
\[
A=QTQ^T,\qquad
T=\begin{bmatrix}T_1&*\\0&T_2\end{bmatrix},
\]
where each diagonal block \(T_i\) is either \(1\times1\) (real eigenvalue) or a real \(2\times2\) block corresponding to a complex conjugate pair.

\subsection{QR factorization}

For \(A\in\mathbb{R}^{m\times n}\) with \(m\ge n\),
\[
A=QR,\qquad Q^TQ=I,\quad R\ \text{upper triangular},\quad R=Q^TA.
\]

\subsection{Eigenvalue perturbation}
Let \(Au=\lambda u\) and \(v^HA=\lambda v^H\) with \(\|u\|_2=\|v\|_2=1\). For \(A(\varepsilon)=A+\varepsilon E\) with \(|\varepsilon|\ll1\), the first–order eigenvalue change is
\[
\delta\lambda=\varepsilon\,v^HEu,\qquad
|\delta\lambda|\le|\varepsilon|\,\|E\|.
\]
Condition number of a simple eigenvalue:
\[
\kappa(\lambda)=\frac{1}{|v^Hu|}.
\]
If \(v^Hu\to0\) (nearly defective), then \(\kappa(\lambda)\to\infty\).

\subsection{Linear system perturbation}

Consider
\[
(A+\varepsilon E)x(\varepsilon)=b+\varepsilon e,\qquad Ax=b.
\]
Let \(\delta x=x(\varepsilon)-x\). Then
\[
(A+\varepsilon E)\,\delta x=\varepsilon(e-Ex),
\qquad
\delta x=\varepsilon(A+\varepsilon E)^{-1}(e-Ex).
\]
Using \((I+\varepsilon A^{-1}E)^{-1}=I-\varepsilon A^{-1}E+O(\varepsilon^2)\),
\[
\delta x=\varepsilon A^{-1}(e-Ex)+O(\varepsilon^2).
\]
Relative error bound:
\[
\frac{\|\delta x\|}{\|x\|}\ \lesssim\ |\varepsilon|\,\kappa(A)\!
\left(\frac{\|e\|}{\|b\|}+\frac{\|E\|}{\|A\|}\right),
\qquad
\kappa(A)=\|A\|\,\|A^{-1}\|.
\]

\subsection{Projection methods}

A projector \(P:\mathbb{C}^n\to\mathbb{C}^n\) satisfies \(P^2=P\).
Then \(\operatorname{Range}(P)=M\) and \(\operatorname{Range}(I-P)=\ker(P)\).

\paragraph{Oblique projection:}
Let \(M=\operatorname{span}\{v_1,\ldots,v_m\}\) and \(W=\operatorname{span}\{w_1,\ldots,w_m\}\).
With \(V=[v_1,\ldots,v_m]\) and \(W=[w_1,\ldots,w_m]\),
\[
P=V\,(W^*V)^{-1}W^*,\qquad Px\in M,\quad W^*(x-Px)=0.
\]

\paragraph{Orthogonal projection:}
Take \(W=V\). Then
\[
P_M=V\,(V^*V)^{-1}V^*,\qquad P_M^*=P_M,\quad P_M^2=P_M,
\]
and the best-approximation property holds:
\[
\|x-P_Mx\|_2=\min_{y\in M}\|x-y\|_2.
\]
