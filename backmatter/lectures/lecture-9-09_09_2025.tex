\section{Lecture 9: 09.09.2025}

\subsection{Krylov Subspace Methods (Saad Ch. 6)}
\paragraph{Motivation:}
Solve $A\mathbf{x} = \mathbf{b}$ for $\mathbf{x}, \mathbf{b} \in \mathbb{R}^n$.

\paragraph{Projection Methods:}
Given $\mathbf{x}_0$ (initial guess), define the residual $\mathbf{r}_0 = \mathbf{b} - A\mathbf{x}_0$.
Choose $\mathcal{K}$ and $\mathcal{L}$ subspaces (same dimension) where you want to find

\[
    \tilde{\mathbf{x}} - \mathbf{x}_0 \in \mathcal{K}, \quad \text{and} \quad \mathbf{b} - A\tilde{\mathbf{x}} \perp \mathcal{L}.
\]

One-dimensional methods: (SD, MR)

\[
    \mathbf{x}_{k+1} = \mathbf{x}_k + \alpha_k \mathbf{r}_k, \quad \mathbf{r}_k = \mathbf{b} - A\mathbf{x}_k.
\]

\begin{align*}
    \mathbf{x}_{1} & = \mathbf{x}_0 + \alpha_0 \mathbf{r}_0, \quad \mathbf{r}_1 = \mathbf{b} - A\mathbf{x}_1 = \mathbf{r}_0 - \alpha_0 A\mathbf{r}_0                      \\
    \mathbf{x}_{2} & = \mathbf{x}_1 + \alpha_1 \mathbf{r}_1, \quad \mathbf{r}_2 = \mathbf{b} - A\mathbf{x}_2 = \mathbf{r}_1 - \alpha_1 A\mathbf{r}_1                      \\
                   & \vdots                                                                                                                                               \\
    \mathbf{x}_{k} & = \mathbf{x}_0 + \tilde{\alpha}_0 \mathbf{r}_0 + \tilde{\alpha}_1 A\mathbf{r}_0 + \ldots + \tilde{\alpha}_{k-1} A^{k-1}\mathbf{r}_0,                 \\
                   & = \mathbf{x}_0 + q_{k-1}(A)\mathbf{r}_0
    \\q_{k-1} &\in \mathbb{P}_{k-1} \\
    \mathbf{x}_{k} & \in \mathbf{x}_0 + \operatorname{span}\{\mathbf{r}_0, A\mathbf{r}_0, \ldots, A^{k-1}\mathbf{r}_0\} =: \mathbf{x}_0 + \mathcal{K}_k(A, \mathbf{r}_0).
\end{align*}

We now define the \textbf{Krylov subspace}:
\begin{definition}{Krylov Subspace}{krylov-subspace}
    Given $A: \mathbb{R}^n \to \mathbb{R}^n$ and $\mathbf{v} \in \mathbb{R}^n$, the $m$-th Krylov subspace is
    \[
        \mathcal{K}_m(A, \mathbf{v}) := \operatorname{span}\{\mathbf{v}, A\mathbf{v}, A^2\mathbf{v}, \ldots, A^{m-1}\mathbf{v}\} = \mathcal{K}_m.
    \]
    Note that $\dim(\mathcal{K}_k(A, \mathbf{v})) \leq k$ and $\dim(\mathcal{K}_k(A, \mathbf{v})) \leq n$.
\end{definition}

\subsection{Important Properties of Krylov Subspaces}
\textbf{1st Property:} What is the smallest $m$ s.t. $A\mathcal{K}_m = \mathcal{K}_m$? (i.e. $\mathcal{K}_m$ is invariant under $A$ meaning $A\mathbf{v} \in \mathcal{K}_m$ for all $\mathbf{v} \in \mathcal{K}_m$)
\begin{definition}{minimal polynomial}{min-poly}
    The minimal polynomial of $\mathbf{v}$ with respect to $A$ is the monic polynomial of the lowest possible degree s.t.
    \[
        A^\mu \mathbf{v} + \sum_{i=0}^{\mu-1} d_i A^i \mathbf{v} = p_A(A)\mathbf{v} = 0.
    \]
    $\mu$ is the grade of $\mathbf{v}$ with respect to $A$.
\end{definition}

\begin{example}{}{}
    Let $A = \begin{pmatrix}
            1 & 1 \\
            0 & 1
        \end{pmatrix}$,
    $\mathbf{v}_1 = \begin{pmatrix}
            1 \\ 1
        \end{pmatrix}$ and $\mathbf{v}_2 = \begin{pmatrix}
            1 \\ 0
        \end{pmatrix}$.
    \[
        A\mathbf{v}_1 = \begin{pmatrix}
            2 \\ 1
        \end{pmatrix}, \quad A^2\mathbf{v}_1 = \begin{pmatrix}
            3 \\ 1
        \end{pmatrix}, \quad A\mathbf{v}_2 = \begin{pmatrix}
            1 \\ 0
        \end{pmatrix}, \quad A^2\mathbf{v}_2 = \begin{pmatrix}
            1 \\ 0
        \end{pmatrix}.
    \]
    Then $\operatorname{grade}(\mathbf{v}_1) = 2$ and $\operatorname{grade}(\mathbf{v}_2) = 1$.
\end{example}

\textbf{2nd Property:}
is that $\operatorname{grade}(\mathbf{v}) \leq n$ where $\mu = \operatorname{grade}(\mathbf{v})$ and $n$ is the size of the matrix $A$.

\subsection{Cayley-Hamilton Theorem}
\begin{theorem}{Cayley-Hamilton Theorem}{}
    Let $A \in \mathbb{R}^{n \times n}$ and
    \[
        p_A(\lambda) = \det(\lambda I - A), \quad p_A \in \mathbb{P}_n
    \]
    be the characteristic polynomial of $A$. Then
    \[
        p_A(A) = 0.
    \]
\end{theorem}

Assume $\mathbf{x} \in \mathcal{K}_m(A, \mathbf{v})$, where $m \geq \mu = \operatorname{grade}(\mathbf{v})$. Then
\begin{align*}
    \mathbf{x} & = q_{m-1}(A)\mathbf{v}, \quad q_{m-1} \in \mathbb{P}_{m-1}                                                                   \\
    q(t)       & = q_1(t) p_A(t) + q_2(t), \qquad p_A \in \mathbb{P}_{\mu}, \, q_2 \in \mathbb{P}_{\mu-1}                            \\
    \mathbf{x} & = q_{m-1}(A)\mathbf{v}\\
    &= q_1(A) p_A(A) \mathbf{v} + q_2(A) \mathbf{v} \\
    &= q_2(A) \mathbf{v}
\end{align*}

\textbf{3rd Property:}
If $\mu = \operatorname{grade}(\mathbf{v})$, then
\[
    A\mathcal{K}_\mu = \mathcal{K}_\mu, \quad \text{and} \quad \mathcal{K}_m = \mathcal{K}_\mu \, \forall m \geq \mu.
\]

\textbf{4th Property:}
\[
    \dim(\mathcal{K}_m) = \min(m, \operatorname{grade}(\mathbf{v})).
\]

If
\begin{align*}
    \dim(\mathcal{K}_m) & = \dim(\mathcal{L}_m)
    \begin{cases}
        \tilde{\mathbf{x}}               & \in \mathbf{x} + \mathcal{K}_m \\
        \mathbf{b} - A\tilde{\mathbf{x}} & \perp \mathcal{L}_m
    \end{cases}
\end{align*}
For simplicity, let $\mathbf{x}_0 = 0$.
If $A\mathcal{K}_m = \mathcal{K}_m$, and $\mathbf{b} \in \mathcal{K}_m$, then the exact solution $\mathbf{x}_\star = \tilde{\mathbf{x}}$ (independent of $\mathcal{L}_m$)\footnote{see lemma 1.36 in Saad}.
\begin{proof}
    Let $\tilde{\mathbf{x}} \in \mathcal{K}$, $A\tilde{\mathbf{x}} \in \mathcal{K}$, and $\mathbf{b} \in \mathcal{K} = A\mathcal{K}$.
    Then
    \begin{align*}
        \mathbf{b} - A\tilde{\mathbf{x}}   & \in \mathcal{K}                                                                                             \\
        \mathbf{b} - A\tilde{\mathbf{x}}   & \perp \mathcal{L}                                                                                           \\
        \mathbf{b} - A\tilde{\mathbf{x}}   & \in \mathcal{K} \cap \mathcal{L}^\perp = \{0\} \quad \Rightarrow \quad \mathbf{b} - A\tilde{\mathbf{x}} = 0 \\
        \Leftrightarrow \tilde{\mathbf{x}} & = \mathbf{x}_\star
    \end{align*}
    \qed
\end{proof}

\begin{lemma}{Lemma 1.36}{}
    Given two subspaces $M$ and $L$ of the same dimension $m$, the following two conditions are mathematically equivalent.

    \begin{enumerate}
        \item No nonzero vector of $M$ is orthogonal to $L$;
        \item For any $x \in \mathbb{C}^n$ there is a unique vector $u$ which satisfies the conditions:
              \begin{equation}
                  u     \in M \quad  x - u  \perp L
              \end{equation}
    \end{enumerate}
\end{lemma}

\subsection{Practical implementation of Krylov Subspace Methods}
Let
\begin{align*}
    A\mathbf{x}                    & = \mathbf{b}, \quad \exists \mathbf{x}_0, \quad \mathbf{r}_0 = \mathbf{b} - A\mathbf{x}_0 \\
    \mathcal{K}_m(A, \mathbf{r}_0) & = \operatorname{span}\{\mathbf{r}_0, A\mathbf{r}_0, \ldots, A^{m-1}\mathbf{r}_0\}
\end{align*}

\subsubsection{FOM: Full Orthogonalization Method}
\begin{align*}
    \mathcal{K}  & = \mathcal{K}_m(A, \mathbf{r}_0)                                            \\
    \mathcal{L}  & = \mathcal{K}                                                               \\
    \mathbf{x}_m & = \mathbf{x}_0 + V_m \left(V_m^\top A V_m\right)^{-1} V_m^\top \mathbf{r}_0 \\
    V_m          & = [\mathbf{v}_1, \ldots, \mathbf{v}_m] \text{ with } V_m^\top V_m = I
\end{align*}

\begin{enumerate}
    \item How to find an orthogonal basis for $\mathcal{K}_m$?
    \item What is $V_m^\top A V_m$?
    \item When to stop?
          \[ \|\mathbf{r}_m\|_2 \leq \text{tol} \]
\end{enumerate}

\subsubsection{1. Arnoldi Algorithm}
\begin{algorithm}[htbp]
    \begin{algorithmic}[0]
        \Require
        \State $A \in \mathbb{R}^{n \times n}$
        \State $\mathbf{v}_1 = \frac{\mathbf{r}_0}{\|\mathbf{r}_0\|_2}$
        \Ensure
        \State $V_{m+1} = [\mathbf{v}_1, \ldots, \mathbf{v}_{m+1}]$ orthonormal basis for $\mathcal{K}_{m+1}(A, \mathbf{r}_0)$
        \State $\overline{H}_m = (h_{ij}) \in \mathbb{R}^{(m+1) \times m}$ upper Hessenberg matrix
        \For{$j = 1, 2, \ldots, m$}
        \State Compute $w = A\mathbf{v}_j$
        \For{$i = 1,\ldots,j$}
        \State $h_{ij} = \langle w, \mathbf{v}_i \rangle$
        \State $w = w - h_{ij}\mathbf{v}_i$
        \EndFor
        \State $h_{j+1,j} = \|w\|_2$
        \If{$h_{j+1,j} = 0$}
        \State Stop (breakdown)
        \EndIf
        \State $\mathbf{v}_{j+1} = w / h_{j+1,j}$
        \EndFor
    \end{algorithmic}
    \caption{Arnoldi Algorithm}
    \label{alg:arnoldi-lecture}
\end{algorithm}
\paragraph{What do we get from the Arnoldi algorithm?}
\begin{align*}
    V_{m+1}        & = [\mathbf{v}_1, \ldots, \mathbf{v}_{m+1}] \in \mathbb{R}^{n \times (m+1)}, \quad V_m = [\mathbf{v}_1, \ldots, \mathbf{v}_m] \in \mathbb{R}^{n \times m} \\
    \overline{H}_m & = (h_{ij}) \in \mathbb{R}^{(m+1) \times m} \text{ upper Hessenberg matrix}, \qquad H_m := \overline{H}_m(1:m,1:m) \in \mathbb{R}^{m\times m}
\end{align*}
s.t.
\begin{align*}
    A V_m &= V_{m+1} \overline{H}_m = V_m H_m + h_{m+1,m} \mathbf{v}_{m+1}\mathbf{e}_m^\top,\\
    V_m^\top A V_m &= H_m.
\end{align*}

Using the Galerkin condition for FOM (take $\mathcal{L}=\mathcal{K}_m$) we obtain the small system
\[
    H_m \mathbf{y}_m = V_m^\top \mathbf{r}_0 = \beta \mathbf{e}_1, \qquad \beta=\|\mathbf{r}_0\|_2,
\]
so
\[
    \mathbf{x}_m = \mathbf{x}_0 + V_m \mathbf{y}_m, \qquad \mathbf{y}_m = H_m^{-1} (\beta \mathbf{e}_1).
\]

The residual can be computed cheaply from the Arnoldi relation:
\begin{align*}
    \mathbf{r}_m &= \mathbf{r}_0 - A V_m \mathbf{y}_m
    = \beta \mathbf{v}_1 - V_{m+1}\overline{H}_m \mathbf{y}_m \\
    &= \beta \mathbf{v}_1 - V_m H_m \mathbf{y}_m - h_{m+1,m}\mathbf{v}_{m+1} \mathbf{e}_m^\top \mathbf{y}_m
    = -h_{m+1,m}\mathbf{v}_{m+1} \mathbf{e}_m^\top \mathbf{y}_m,
\end{align*}
since $H_m\mathbf{y}_m=\beta\mathbf{e}_1$. Hence
\[
    \|\mathbf{r}_m\|_2 = |h_{m+1,m}|\,|\mathbf{e}_m^\top \mathbf{y}_m|.
\]

Thus we get the FOM algorithm (Arnoldi performed incrementally; solve the small system at each step and check residual):

\begin{algorithm}[H]
    \begin{algorithmic}[0]
        \Require
        \State $A \in \mathbb{R}^{n \times n}, \mathbf{b} \in \mathbb{R}^n, \mathbf{x}_0 \in \mathbb{R}^n, m_{\max} \in \mathbb{N}, \text{tol} > 0$
        \State $\mathbf{r}_0 = \mathbf{b} - A\mathbf{x}_0, \quad \beta = \|\mathbf{r}_0\|_2$
        \State $\mathbf{v}_1 = \mathbf{r}_0 / \beta$
        \Ensure
        \State $\mathbf{x}_j$ approximations, stop when converged or breakdown
        \For{$j = 1, 2, \ldots, m_{\max}$}
        \State Perform one Arnoldi step to compute $h_{1:j+1,j}$ and $\mathbf{v}_{j+1}$ (see Alg. \ref{alg:arnoldi-lecture})
        \State Let $H_j = \overline{H}_j(1:j,1:j)$ and $V_j = [\mathbf{v}_1,\ldots,\mathbf{v}_j]$
        \State Solve $H_j \mathbf{y}_j = \beta \mathbf{e}_1$
        \State $\mathbf{x}_j = \mathbf{x}_0 + V_j \mathbf{y}_j$
        \State $\mathbf{r}_j = -h_{j+1,j} \mathbf{v}_{j+1} \mathbf{e}_j^\top \mathbf{y}_j$
        \If{$\|\mathbf{r}_j\|_2 \leq \text{tol}$}
        \State Return $\mathbf{x}_j$
        \EndIf
        \If{$h_{j+1,j}=0$}
        \State Breakdown: exact solution in $\mathcal{K}_j$ (stop)
        \EndIf
        \EndFor
    \end{algorithmic}
    \caption{Full Orthogonalization Method (FOM)}
    \label{alg:fom}
\end{algorithm}